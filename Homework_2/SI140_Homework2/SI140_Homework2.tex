%%%%%%%%%%%%%%%%%%%%%%%%%%%%%%%%%%%%%%%%%%%%%%%%%%%%%%%%%%%%%%%%%%%%%%%%%%%%%%%%%%%%
%Do not alter this block of commands.  If you're proficient at LaTeX, you may include additional packages, create macros, etc. immediately below this block of commands, but make sure to NOT alter the header, margin, and comment settings here. 
\documentclass[12pt]{article}
 \usepackage[margin=1in]{geometry} 
 \usepackage{mathrsfs}
\usepackage{amsmath,amsthm,amssymb,amsfonts, enumitem, fancyhdr, color, comment, graphicx, environ}
\pagestyle{fancy}
\setlength{\headheight}{65pt}
\newenvironment{problem}[2][Problem]{\begin{trivlist}
\item[\hskip \labelsep {\bfseries #1}\hskip \labelsep {\bfseries #2.}]}{\end{trivlist}}
\newenvironment{sol}
    {\emph{Solution:}
    }
    {
    \qed
    }
\specialcomment{com}{ \color{blue} \textbf{Comment:} }{\color{black}} %for instructor comments while grading
\NewEnviron{probscore}{\marginpar{ \color{blue} \tiny Problem Score: \BODY \color{black} }}
%%%%%%%%%%%%%%%%%%%%%%%%%%%%%%%%%%%%%%%%%%%%%%%%%%%%%%%%%%%%%%%%%%%%%%%%%%%%%%%%%
\usepackage[UTF8]{ctex}




%%%%%%%%%%%%%%%%%%%%%%%%%%%%%%%%%%%%%%%%%%%%%
%Fill in the appropriate information below
\lhead{Name: 陈稼霖\\ StudentID: 45875852}  %replace with your name
\rhead{SI 140 \\ Probability and Statistics \\ Semester Spring 2019 \\ Assignment 2} %replace XYZ with the homework course number, semester (e.g. ``Spring 2019"), and assignment number.
%%%%%%%%%%%%%%%%%%%%%%%%%%%%%%%%%%%%%%%%%%%%%


%%%%%%%%%%%%%%%%%%%%%%%%%%%%%%%%%%%%%%
%Do not alter this block.
\begin{document}
%%%%%%%%%%%%%%%%%%%%%%%%%%%%%%%%%%%%%%


%Solutions to problems go below.  Please follow the guidelines from https://www.overleaf.com/read/sfbcjxcgsnsk/


%Copy the following block of text for each problem in the assignment.
\begin{problem}{1} 
Show that if $P$ and $Q$ are two probability measures defined on the same (countable) sample space, then $aP+bQ$ is also a probability measure for any two nonnegative numbers $a$ and $b$ satisfying $a+b=1$. Give a concrete illustration of such a mixture.
\end{problem}
\begin{sol}
Let $\Omega$ be the sample space that $P$ and $Q$ are defined on and $\mathscr{F}\subset2^{\Omega}$ be a $\sigma$-algebra. Because $P$ and $Q$ are two probability measures, for every set $A\in\mathscr{F}$,
\[
P(A)\geq0,~~Q(A)\geq0
\]
And because $a$ and $b$ are two nonnegative number,
\begin{equation}
\label{1.1}
aP(A)+bQ(A)\geq0
\end{equation}
Because $A$ and $B$ are two probability measures, for any countable collections of disjoint sets $A_1, A_2, ...\in\mathscr{F}$,
\[
P(\bigcup_{j=1}^{\infty}A_j)=\sum_{j=1}^{\infty}P(A_j),~~Q(\bigcup_{j=1}^{\infty}A_j)=\sum_{j=1}^{\infty}Q(A_j)
\]
we have
\begin{equation}
\label{1.2}
\begin{split}
aP(\bigcup_{j=1}^{\infty}A_j)+bQ(\bigcup_{j=1}^{\infty}A_j)=&a\sum_{j=1}^{\infty}P(A_j)+b\sum_{j=1}^{\infty}Q(A_j)\\
=&\sum_{j=1}^{\infty}[aP(A_j)+bQ(A_j)]
\end{split}
\end{equation}
Because $P$ and $Q$ are two probability measures,
\[
P(\Omega)=1,~~P(\Omega)=1
\]
And because $a+b=1$,
\begin{equation}
\label{1.3}
aP(\Omega)+bQ(\Omega)=a\times1+b\times1=1
\end{equation}
Because of (\ref{1.1}), (\ref{1.2}) and (\ref{1.3}), $aP+bQ$ is also a probability measure.

A concrete illustration of such a mixture: Suppose $P$ and $Q$ are two probability measures defined on the same sample space $\Omega=\{0,1\}$
\[
P(A)=\left\{\begin{array}{ll}
0,&A=\emptyset\\
0.6,&\text{if }A=\{0\}\\
0.4,&\text{if }A=\{1\}\\
1,&\text{if }A=\Omega
\end{array}\right.,~~
Q(A)=\left\{\begin{array}{ll}
0,&\text{if }A=\emptyset\\
0.4,&\text{if }A=\{0\}\\
0.6,&\text{if }A=\{1\}\\
1,&\text{if }A=\Omega
\end{array}\right.
\]
and
\[
a=0.5,~~b=0.5
\]
Then we have
\[
aP(A)+bQ(A)=\left\{\begin{array}{ll}
0,&\text{if }A=\emptyset\\
0.5,&\text{if }A=\{0\}\\
0.5,&\text{if }A=\{1\}\\
1,&\text{if }A=\Omega
\end{array}\right.
\]
It is obvious that $aP+bQ$ satisfies the three axioms
\begin{enumerate}
\item For every set $A\in\mathscr{F}$, $aP(A)+bQ(A)\geq0$
\item For any collections of disjoint sets $A_1, A_2, ...\in\mathscr{F}$, $$aP(\bigcup_{j=1}^{\infty}A_j)+bQ(\bigcup_{j=1}^{\infty}A_j)=\sum_{j=1}^{\infty}[aP(A_j)+bQ(A_j)]$$
\item $aP(\Omega)+bQ(\Omega)=1$
\end{enumerate}
Therefore, $aP+bQ$ is also a probability measure.
\end{sol}

%Copy the following block of text for each problem in the assignment.
\begin{problem}{2}
If $P$ is a probability measure, show that the function $P/2$ satisfies Axioms (i) and (ii) but not (iii). The function $P^2$ satisfies (i) and (iii) but not necessarily (ii); give a conterexample to (ii).
\end{problem}
\begin{sol}
Let $\Omega$ be the sample space and $\mathscr{F}\in2^{\Omega}$ be a $\sigma$-algebra. Because $P$ is a probability measure, it satisfies the three axioms:
\begin{enumerate}
\item For every set $A\in\mathscr{F}$, $P(A)\geq0$
\item For any collections of disjoint sets $A_1, A_2, ...\in\mathscr{F}$, $$P(\bigcup_{j=1}^{\infty}A_j)=\sum_{j=1}^{\infty}P(A_j)$$
\item $P(\Omega)=1$
\end{enumerate}

Because of 1., for every set $A\in\mathscr{F}$, $\frac{P(A)}{2}\geq0$. And because of 2., for any collections of disjoint sets $A_1, A_2, ...\in\mathscr{F}$, 
\[
\frac{P(\bigcup_{j=1}^{\infty}A_j)}{2}=\frac{\sum_{j=1}^{\infty}P(A_j)}{2}=\sum_{j=1}^{\infty}\frac{P(A_j)}{2}
\]
However,
\[
\frac{P(\Omega)}{2}=\frac{1}{2}\neq1
\]
Therefore, the function $P/2$ satisfies Axioms (i) and (ii) but not (iii).\\
It is obvious that, for every set $A$, $P^2(A)\geq0$. And because of 3.,
\[
P^2(\Omega)=1^2=1
\]
However, for a collection of disjoint sets $A_1, A_2, ...\in\mathscr{F}$,
\[
P^2(\bigcup_{j=1}^{\infty}A_j)=[\sum_{j=1}^{\infty}P(A_j)]^2
\]
which does not necessarily equal $\sum_{j=1}^{\infty}P^2(A_j)$.\\
A counterexample: sets $A_1, A_2, ...$ are disjoint, $P(A_1)=P(A_2)=\frac{1}{2}$, $P(A_j)=0$ for $j\geq3$, $$P^2(\bigcup_{j=1}^{\infty}A_j)=[P(A_1)+P(A_2)]^2=1\neq\frac{1}{2}=P^2(A_1)+P^2(A_2)=\sum_{j=1}^{\infty}P^2(A_j)$$\\
Therefore, the function $P^2$ satisfies (i) and (iii) but not necessarily (ii).
\end{sol}



%Copy the following block of text for each problem in the assignment.
\begin{problem}{3}
Show that if the two events $(A,B)$ are independent, then so are $(A,B^c)$, $(A^c,B)$ and $(A^c,B^c)$. Generalize this result to three independent events. 
\end{problem}
\begin{sol}
Because the two events $(A,B)$ are independent,
\[
P(A\cap B)=P(A)P(B)
\]
Then we have
\begin{align*}
P(A\cap B^c)=&P(A\cap(\Omega-B))\\
=&P(A\cap\Omega-A\cap B)\\
=&P(A-A\cap B)\\
&\text{(because $(A\cap B)\subset A$)}\\
=&P(A)-P(A\cap B)\\
=&P(A)-P(A)P(B)\\
=&P(A)[1-P(B)]\\
=&P(A)P(B^c)
\end{align*}
$\Longrightarrow$ the two events $(A,B^c)$ are independent. Similarly, so are $(A^c,B)$.\\
And
\begin{align*}
P(A^c\cap B^c)=&P(A^c\cap(\Omega-B))\\
=&P(A^c-A^c\cap B)\\
=&P(A^c)-P(A^c\cap B)\\
=&P(A^c)-P(A^c)P(B)\\
=&P(A^c)[1-P(B)]\\
=&P(A^c)P(B^c)
\end{align*}
$\Longrightarrow$ the two events $(A^c,B^c)$ are independent.\\
Generalize to this result to three independent events: If the three events $(A,B,C)$ are independent, then
\begin{gather*}
P(A\cap B)=P(A)P(B),~~P(B\cap C)=P(B)P(C),~~P(C\cap A)=P(C)P(A)\\
P(A\cap B\cap C)=P(A)P(B)P(C)
\end{gather*}
Then we have
\begin{align*}
P(A\cap B\cap C^c)=&P(A\cap B\cap(\Omega-C))\\
=&P(A\cap B-A\cap B\cap C)\\
=&P(A\cap B)-P(A\cap B\cap C)\\
=&P(A)P(B)-P(A)P(B)P(C)\\
=&P(A)P(B)[1-P(C)]\\
=&P(A)P(B)P(C^c)
\end{align*}
\[
P(A\cap B)=P(A)P(B),~~P(B\cap C^c)=P(B)P(C^c),~~P(C^c\cap A)=P(C^c)P(A)
\]
$\Longrightarrow$the three events $(A,B,C^c)$ are independent. Similarly, so are $(A,B^c,C)$ and $(A^c,B,C)$.
\begin{align*}
P(A\cap B^c\cap C^c)=&P(A\cap B^c\cap(\Omega-C))\\
=&P(A\cap B^c-A\cap B^c\cap C)\\
=&P(A\cap B^c)-P(A\cap B^c\cap C)\\
=&P(A)P(B^c)-P(A)P(B^c)P(C)\\
=&P(A)P(B^c)[1-P(C)]\\
=&P(A)P(B^c)P(C^c)
\end{align*}
\[
P(A\cap B^c)=P(A)P(B^c),~~P(B^c\cap C^c)=P(B^c)P(C^c),~~P(C^c\cap A)=P(C^c)P(A)
\]
$\Longrightarrow$the three events $(A,B^c,C^c)$ are independent. Similarly, so are $(A^c,B,C^c)$ and $(A^c,B^c,C)$.
\begin{align*}
P(A^c\cap B^c\cap C^c)=&P(A^c\cap B^c\cap(\Omega-C))\\
=&P(A^c\cap B^c-A^c\cap B^c\cap C)\\
=&P(A^c\cap B^c)-P(A^c\cap B^c\cap C)\\
=&P(A^c)P(B^c)-P(A^c)P(B^c)P(C)\\
=&P(A^c)P(B^c)[1-P(C)]\\
=&P(A^c)P(B^c)P(C^c)
\end{align*}
\[
P(A^c\cap B^c)=P(A^c)P(B^c),~~P(B^c\cap C^c)=P(B^c)P(C^c),~~P(C^c\cap A^c)=P(C^c)P(A^c)
\]
$\Longrightarrow$the three events $(A^c,B^c,C^c)$ are independent.
\end{sol}



%Copy the following block of text for each problem in the assignment.
\begin{problem}{4}
Show that if $A,B,C$ are independent events, then $A$ and $B\cup C$ are independent, and $A\setminus B$ and $C$ are independent.
\end{problem}
\begin{sol}
Because $A, B, C$ are independent,
\begin{gather*}
P(A\cap B)=P(A)P(B),~~P(B\cap C)=P(B)P(C),~~P(C\cap A)=P(C)P(A)\\
P(A\cap B\cap C)=P(A)P(B)P(C)
\end{gather*}
Then we have
\begin{align*}
P(A\cap(B\cup C))=&P((A\cap B)\cup(A\cap C))\\
=&P(A\cap B)+P(A\cap C)-P((A\cap B)\cap(A\cap C))\\
=&P(A\cap B)+P(A\cap C)-P(A\cap B\cap C)\\
=&P(A)P(B)+P(A)P(C)-P(A)P(B)P(C)\\
=&P(A)[P(B)+P(C)-P(B)P(C)]\\
=&P(A)[P(B)+P(C)-P(B\cap C)]\\
=&P(A)P(B\cup C)
\end{align*}
$\Longrightarrow$ $A$ and $B\cup C$ are independent.\\
And
\begin{align*}
P((A\setminus B)\cap C)=&P(A\cap B^c\cap C)\\
&\text{(according to the conclusions of Problem 3)}\\
=&P(A)P(B^c)P(C)\\
=&P(A\cap B^c)P(C)\\
=&P(A\setminus B)P(C)
\end{align*}
$\Longrightarrow$ $A\setminus B$ and $C$ are independent.
\end{sol}



%Copy the following block of text for each problem in the assignment.
\begin{problem}{5}
Let $\Omega$ be a set and $\mathscr{F}\subset2^{\Omega}$ be a $\sigma$-algebra. A function $P$: $\mathscr{F}\to \mathbb{R}\cup \{+\infty,-\infty\}$ is called a probability measure if it satisfies the following three properties:
\begin{enumerate}
    \item For all $A\in \mathscr{F}$, $P(A)\geq 0$
    \item $P(\Omega)=1$
    \item For all countable collections disjoint $A_1,A_2,...$ in $\mathscr{F}$,
    \[P(\bigcup_{j=1}^{\infty}A_j)=\sum_{j=1}^{\infty} P(A_j)\]
\end{enumerate}
Given a nested increasing sequence of events $A_1\subset A_2\subset A_3 ... \subset A_n \subset ...$ such that $\cup_{i=1}^{\infty}A_i$ is also an event, prove that $$\lim_{n\to \infty}P(A_n)=P(\bigcup_{i=1}^{\infty}A_i)$$
\end{problem}
\begin{sol}
Because $A_1\subset A_2\subset A_3 ... \subset A_n \subset ...$, sets $A_1, (A_2-A_1), (A_3-A_2), ...$ are disjoint.
\begin{align*}
P(\bigcup_{i=1}^{\infty}A_i)=&P(A_1+\bigcup_{i=2}^{\infty}(A_i-A_{i-1}))\\
=&P(A_1)+\sum_{i=2}^{\infty}[P(A_i)-P(A_{i-1})]\\
=&\lim_{n\to\infty}\{P(A_1)+\sum_{i=2}^n[P(A_i)-P(A_{i-1})]\}\\
=&\lim_{n\to\infty}P(A_1\cap\bigcap_{i=1}^n(A_i-A_{i-1}))\\
=&\lim_{n\to\infty}P(A_n)
\end{align*}
\end{sol}



%Copy the following block of text for each problem in the assignment.
\begin{problem}{6}
Find an example where $$P(AB) < P(A)P(B)$$

\end{problem}
\begin{sol}
Let $\Omega=\{0,1\}$, $A=\{0\}$, $B=\{1\}$, $P(A)=\frac{1}{2}$, $P(B)=\frac{1}{2}$. Then we have
\[
P(AB)=P(\emptyset)=0<\frac{1}{4}=\frac{1}{2}\times\frac{1}{2}=P(A)P(B)
\]
\end{sol}


%Copy the following block of text for each problem in the assignment.
\begin{problem}{7}
What can you say about the event A if it is independent of itself? If
the events A and B are disjoint and independent, what can you say of
them?
\end{problem}
\begin{sol}
If the event $A$ is independent of itself, then
\begin{gather*}
P(A)=P(A\cap A)=P(A)P(A)\\
\Longrightarrow P(A)=0\text{ or }P(A)=1
\end{gather*}
If the events $A$ and $B$ are disjoint and independent, then
\begin{gather*}
P(A)P(B)=P(A\cap B)=P(A)+P(B)-P(A\cup B)=0\\
\Longrightarrow P(A)=0\text{ or }P(B)=0
\end{gather*}
\end{sol}

%Copy the following block of text for each problem in the assignment.
\begin{problem}{8}
Prove that $$P(A\cup B\cup C) = P(A) + P(B) + P(C) - P(AB) - P(AC) - P(BC) + P(ABC)$$
when A, B, C are independent by considering $P(A^{c}B^{c}C^{c})$
\end{problem}
\begin{sol}
\begin{align*}
P(A\cup B\cup C)=&P(\Omega-A^cB^cC^c)\\
=&P(\Omega)-P(A^cB^cC^c)\\
&\text{(according to the conclusions of Problem 3)}\\
=&1-P(A^c)P(B^c)P(C^c)\\
=&1-[1-P(A)][1-P(B)][1-P(C)]\\
=&P(A)+P(B)+P(C)-P(A)P(B)-P(A)P(C)-P(B)P(C)\\
&+P(A)P(B)P(C)\\
=&P(A)+P(B)+P(C)-P(AB)-P(AC)-P(BC)+P(ABC)
\end{align*}
\end{sol}


%Copy the following block of text for each problem in the assignment.
\begin{problem}{9}
Let $S = (-\infty,+\infty)$, the real line. Then $\mathscr{F}$ is chosen to contain all sets of the form 
$$(a,b], [a,b], [a,b), (a,b)$$
for all real numbers a and b. (Unions of these form are in $\mathscr{F}$.) Show that  $\mathscr{F}$ is a Borel field.
\end{problem}
\begin{sol}
$\mathscr{F}\subset2^S$. For any set $(a,b)$ s.t. $a$ and $b$ are real numbers, its complement
\begin{equation}
\label{9.1}
(-\infty,a]\cup[b,+\infty)=(\bigcup_{i=1}^{\infty}(a-i,a])\cup(\bigcup_{i=1}^{\infty}[b,b+i))\in\mathscr{F}
\end{equation}
Similarly, the complements of $(a,b]$, $[a,b]$ and $[a,b)$ are also in $\mathscr{F}$.\\
And according to the problem description, unions of these four forms are in $\mathscr{F}$, so for $A_1,A_2, ...$ in $\mathscr{F}$,
\begin{equation}
\label{9.2}
\bigcup_{j=1}^{\infty}A_j\in\mathscr{F}
\end{equation}
Because of (\ref{9.1}) and (\ref{9.2}), $\mathscr{F}$ is a Borel field.
\end{sol}

%Copy the following block of text for each problem in the assignment.
\begin{problem}{10}
Suppose that the land of a square kingdom is divided into three strips
A, B, C of equal area and suppose the value per unit is in the ratio
of 1:3:2. For any piece of (measurable) land $S$ in this kingdom, the
relative value with respect to that of the kingdom is then given by the
formula:
$$V(S) = \frac{P(SA)+3P(SB)+2P(SC)}{2}$$
where P is as in Example 2 of 2.1. Show that V is a probability
measure.
\end{problem}
\begin{sol}
For any piece of land $S$ in this kingdom,
\begin{align*}
V(S)=&\frac{P(SA)+3P(SB)+2P(SC)}{2}\\
=&\frac{|SA|+3|SB|+2|SC|}{2|S|}
\end{align*}
where $|SA|, |SB|, |SC|$ and $|S|$ are area of land pieces $SA, SB, SC$ and $S$, and are all positive, so
\begin{equation}
\label{10.1}
V(S)\geq0
\end{equation}
For any countable collections of disjoint sets $S_1, S_2, ...$ in the kingdom,
\begin{equation}
\label{10.2}
\begin{split}
V(\bigcup_{j=1}^{\infty}S_j)=&\frac{P(A\cap\bigcup_{j=1}^{\infty}S_j)+3P(B\cap\bigcup_{j=1}^{\infty}S_j)+2P(C\cap\bigcup_{j=1}^{\infty}S_j)}{2}\\
=&\frac{P(\bigcup_{j=1}^{\infty}(S_jA))+3P(\bigcup_{j=1}^{\infty}(S_jB))+2P(\bigcup_{j=1}^{\infty}(S_jC))}{2}\\
=&\frac{\sum_{j=1}^{\infty}P(S_jA)+3\sum_{j=1}^{\infty}P(S_jB)+2\sum_{j=1}^{\infty}P(S_jC)}{2}\\
=&\sum_{j=1}^{\infty}\frac{P(S_jA)+3P(S_jB)+2P(S_jC)}{2}\\
=&\sum_{j=1}^{\infty}V(S_j)
\end{split}
\end{equation}
Let $\Omega$ be the total land of this kingdom
\begin{equation}
\label{10.3}
P(\Omega)=\frac{P(A)+3P(B)+2P(C)}{2}=\frac{\frac{1}{3}+3\times\frac{1}{3}+2\times\frac{1}{3}}{2}=1
\end{equation}
Because of (\ref{10.1}), (\ref{10.2}) and (\ref{10.3}), $V$ is a probability measure.
\end{sol}



















































































%%%%%%%%%%%%%%%%%%%%%%%%%%%%%%%%%%%%%%%%
%Do not alter anything below this line.
\end{document}